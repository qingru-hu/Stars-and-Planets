\documentclass[a4paper,12pt]{article}
\usepackage{graphicx, geometry, subfigure, amsmath, adjustbox, array}
\usepackage{xcolor}
\geometry{a4paper,left=2cm,right=2cm,top=1cm,bottom=2cm}
\setlength{\baselineskip}{12pt}
\renewcommand\arraystretch{1.5}
\renewcommand{\d}{\mathrm{d}}
\newcommand{\cm}{\mathrm{cm}}
\newcommand{\s}{\mathrm{s}}
\newcommand{\g}{\mathrm{g}}

\title{\textbf{Stars and Planets Problem Set4}}
\author{Qingru Hu}
\date{\today}

\begin{document}
\maketitle
\section*{\textbf{Exercise \uppercase\expandafter{\romannumeral4}.1 Photon diffusion}}
\subsection*{(a)}
The mean free path of the distribution $\bar{s}$ is:
\begin{align*}
    \bar{s} = \frac{\int_{0}^{+\infty} s l^{-1} e^{-s/l} \d s}{\int_{0}^{+\infty} l^{-1} e^{-s/l} \d s} = 
    \frac{l \int_{0}^{+\infty} x e^{-x} \d x}{\int_{0}^{+\infty} e^{-x} \d x}
    = l \ (\text{where } x = s/l)
\end{align*}


\subsection*{(b)}
% The distribution makes the escape time longer. The probability of the step size falling between $s$, $s+\d s$ 
% decreases exponentially as the step size increases. The probability of the step size smaller than the mean free path $l$
% is much more higher than that of the step size bigger than the mean free path, so the escape time will become longer.
The motion of the photon is:
\begin{align*}
    \vec{S} = \vec{s}_1 + \vec{s}_2 + \vec{s}_3 + \cdots
\end{align*}
The average of the squared length:
\begin{align*}
    <S^2> = (\vec{s}_1 + \vec{s}_2 + \vec{s}_3 + \cdots)(\vec{s}_1 + \vec{s}_2 + \vec{s}_3 + \cdots) 
    = \sum_{i=1}^N \vec{s}_i \cdot \vec{s}_i + \sum_{i=1}^N \sum_{i\neq j}^N \vec{s}_i \vec{s}_j
\end{align*}
The second equals 0 because the symmetry of the random walk, so we have:
\begin{align*}
    <S^2> &= R^2 = \sum_{i=1}^N \vec{s}_i \cdot \vec{s}_i = N <s^2> \\
    <s^2> &= \int_{0}^{+\infty} s^2  l^{-1} e^{-s/l} \d s = 2l^2 \\
    N &= \frac{R^2}{2 l^2}
\end{align*}
Therefore, the escape time will be shorter:
\begin{align*}
    t_\text{escp} = \frac{N l}{c} = \frac{R^2}{2cl}
\end{align*}

\subsection*{(c)}
I think the escape time will become shorter. If there exists density gradient, diffusion will come into the picture.
Diffusion will help photons to move from dense core to sparse surface, making the escape time shorter.



\section*{\textbf{Exercise \uppercase\expandafter{\romannumeral4}.2 Ledoux stability criterion}}
\subsection*{(a)}
The expression for the two gradients are:
\begin{align*}
    \nabla &= \frac{\d \log T}{\d \log P} \\
    \nabla_\mu &= \frac{\d \log \mu }{\d \log P}
\end{align*}
Divide both sides of Equation (3) by $\d P/ p = \d \log P$:
\begin{align*}
    1 &= \chi_\rho \frac{\d \log \rho}{\d \log P} + \chi_T \frac{\d \log T}{\d \log P} + \chi_\mu \frac{\d \log \mu}{\d \log P} \\
    1 &= \chi_\rho \frac{\d \log \rho}{\d \log P} + \chi_T \nabla + \chi_\mu \nabla_\mu
\end{align*}

\subsection*{(b)}
Since for an adiabatic gas the composition is the same, $\nabla_\mu = 0$. 
For constant entropy process, $\nabla = \nabla_\text{ad}$ and $P\propto \rho^{\gamma_{\text{ad}}}, \d \log P/\d \log \rho = \gamma_\text{ad}$.
Equation(3) becomes:
\begin{align*}
    1 &= \chi_\rho / \gamma_\text{ad} + \chi_T \nabla_\text{ad} \\
    \nabla_\text{ad} &= (1-\chi_\rho / \gamma_\text{ad})/\chi_T
\end{align*}

\subsection*{(c)}
For an ideal mono-atomic gas, the equation of state is $P = \frac{\rho kT}{\mu m_u}$ and $\gamma_\text{ad} = 5/3$. Besides, 
$\chi_\rho = 1$ and $\chi_T = 1$.
\begin{align*}
    \nabla_\text{ad} &= (1-\chi_\rho / \gamma_\text{ad})/\chi_T = 2/5
\end{align*}

\subsection*{(d)}
From Equation(4) we can get:
\begin{align*}
    \frac{\d \log \rho}{\d \log P} = (1 - \chi_T \nabla - \chi_\mu \nabla_\mu)/\chi_\rho
\end{align*}
From section (b) we can get:
\begin{align*}
    \frac{1}{\gamma_\text{ad}} = (1 - \chi_T \nabla_\text{ad})/\chi_\rho
\end{align*}
Th e instability criterion can lead to the condition for convection:
\begin{align*}
    \frac{\d \log \rho}{\d \log P} &< \frac{1}{\gamma_\text{ad}} \\
    (1 - \chi_T \nabla - \chi_\mu \nabla_\mu)/\chi_\rho &< (1 - \chi_T \nabla_\text{ad})/\chi_\rho \\
    \nabla > \nabla_\text{ad} -& \frac{\chi_\mu}{\chi_T}\nabla_\mu
\end{align*}


\section*{\textbf{Exercise \uppercase\expandafter{\romannumeral4}.3 Structure of the Sun}}
\subsection*{(a)}
The relation between the mean molecular mass $\mu$ and the fractional radius is shown as below.
\begin{figure}[htbp]
    \centering
    \includegraphics*[width=8cm]{mu.png}
\end{figure}

\subsection*{(b)}
The two dashed lines in the above figure shows the 20\% and 95\% fractional radius. 
We all know that $\mu_H = 1/2$, $\mu_{He}=4/3$ and $\mu_{Z} = 2$. In the central $20\%$, 
the core of the sun is mostly composed of Helium, so the mean molecular weight is between 1/2 and 
4/3; the interior 20\%-95\% region is mostly composed of Hydrogen, so the mean molecular weight is 
near 1/2; in the outer 5\% region, metals have been brought into this shell due to convection, so the 
mean molecular weight increase sharply in the outer shell.

\subsection*{(c)}
The numerical errors occur in regions near the center and the surface. I exclude the first 60 data points 
from r=0 and use median filter to smooth the curve.
\begin{figure}[htbp]
    \centering
    \includegraphics*[width=8cm]{nabla.png}
\end{figure}

\subsection*{(d)}
Assume the composition of the Sun is ideal mono-atomic gas. 
From Exercise2(c) we get that for an ideal mono-atomic gas $\nabla_{\text{ad}} = 2/5=0.4$.
Therefore, we can roughly guess that the outer 30\% region of the Sun is convective.


\section*{\textbf{Exercise \uppercase\expandafter{\romannumeral4}.4 Mass-radius relationships for stars}}
\subsection*{(a)}
Time $M$ to both sides of Equation(8) and use the homology:
\begin{align*}
    \frac{\d L}{\d x} &= \epsilon_0 \rho M T^\nu \\
    \frac{\d L_1}{\d L_2} &= \frac{\rho_1 M_1 T_1^\nu}{\rho_2 M_2 T_2^\nu} \\
    L & \propto M^{\nu+2} R^{-3-\nu} \mu^{\nu}
\end{align*}

\subsection*{(b)}
\begin{align*}
    L &\propto M^3 \nu^4 \\
    L & \propto M^{\nu+2} R^{-3-\nu} \mu^{\nu} \\
    M^{\nu+2} R^{-3-\nu} \mu^{\nu} &\propto M^3 \nu^4 \\
    R &\propto M^{\frac{\nu -1}{\nu +3}} \mu^{\frac{\nu -4}{3+\nu }}
\end{align*}

\subsection*{(c)}
From Equation(9) and the given equations we can have:
\begin{align*}
    T &\propto MR^{-1} \mu \\
    R &\propto M^{\frac{\nu -1}{\nu +3}} \mu^{\frac{\nu -4}{3+\nu }} \\
\end{align*}
Therefore we can get:
\begin{align*}
    T &\propto M^{\frac{4}{3+\nu}} \mu^{\frac{7}{3+\nu}} \\
    M &\propto T^{} \mu ^{-7/4}
\end{align*}
Assume the composition of the star are almost the same, so $\mu$ is a constant.
\begin{align*}
    \frac{M_{\text{min}}}{M_{\bigodot}} \propto (\frac{T_{\text{min}}}{T_{\bigodot}})^{\frac{3+\nu}{4}}
\end{align*}
Plug in $\nu=4$ and $T_{\bigodot} = 2\times 10^7 \ K, T_{\text{min}} = 10^7\ K$, and $M_{\text{min}} = 0.30 M_{\bigodot}$.

\subsection*{(d)}
This answer is not correct. We have observed stars that burn hydrogen with masses smaller than $0.30 M_{\bigodot}$.

\subsection*{(e)}
I think the relation $L \propto M^3 \mu^4$ will not hold for small mass stars. We derive this relation from 
radiative diffusion, but for small mass stars, convection is dominant, so the formula for radiative diffusion 
is not correct for small mass stars.

\section*{\textbf{Exercise \uppercase\expandafter{\romannumeral4}.5 Critical core mass for planets}}
\subsection*{(a)}
For a radiatively supported envelope, assuming the opacity $\kappa$ and luminosity $L$ are constant:
\begin{align*}
    \frac{L}{4 \pi r^2} &= F = -D_{phot} \nabla u_{rad} = - \frac{c}{3} \frac{1}{\kappa \rho} \frac{16\sigma}{c} 4T^3 \frac{\d T}{\d r} \\
    \frac{\d T}{\d r} &= -\frac{3 \kappa \rho L}{64 \pi r^2 \sigma T^3} 
\end{align*}
From the hydrostatic balance, we also have:
\begin{align*}
    \frac{\d P}{\d r} = -\frac{GM_{tot}}{r^2} \rho
\end{align*}
Therefore, the radiation difussion gradient is:
\begin{align*}
    \nabla_{rad} = \frac{\d \log T}{\d \log P} &= \frac{P}{T} \frac{\d T}{\d r} \frac{\d r}{\d P} = \frac{3 \kappa L P}{64 \pi \sigma G M_{tot} T^4} = \frac{WP}{G M_{tot} T^4} \\
    \d(WP) &= \d(GM_{tot} T^4/4) \\
    P &= \frac{GM_{tot}T^4}{4W} 
\end{align*}
where $W = 3\kappa L / 64 \pi \sigma_{sb}$.

\subsection*{(b)}
The ideal gas law and hydrostatic balance are:
\begin{align*}
    P &= \frac{\rho kT}{\mu m_u} \\
    \frac{\d P}{\d r} &= -\frac{GM_{tot}}{r^2} \rho
\end{align*}
Plug in Equation(11) and we can get:
\begin{align*}
    \d T = -A \frac{\d r}{r^2}
\end{align*}
where $A = GM_{tot}\mu m_u / 4k$.
Integrate on both sides and we get:
\begin{align*}
    T  = A \frac{1}{r}
\end{align*}

\subsection*{(c)}
\begin{align*}
    \rho = -\frac{r^2}{G M_{tot}} \frac{\d P}{\d T} \frac{\d T}{\d r} = \frac{A^4}{W}\frac{1}{r^3}
\end{align*}
Integrate this density profile we can get:
\begin{align*}
    M_{env} = \int_{r_{in}}^{r_{out}} \rho 4\pi r^2 \d r = \frac{4\pi}{W} A^4 \Lambda
\end{align*}
where $\Lambda  = \log (r_{out}/r_{in}), A = GM_{tot}\mu m_u / 4k$.


\subsection*{(d)}
\begin{figure}[htbp]
    \centering
    \includegraphics*[width=12cm]{e5.png}
\end{figure}

\subsection*{(e)}
For higher core mass, the envelope mass is also higher. When the envelope mass is comparable to that of the 
core, our second assumption (the gravitational mass interior to r being approximated by the total mass) does not 
hold anymore. We must consider the gravity gradient caused by the envelope mass in our model.

The critical core mass gives us a rough mass division between rocky planets with thin atmosphere and gas giants.

\subsection*{(f)}
Higher luminosity provides higher radiation pressure. For a given total mass, the core mass will comprise more and the 
envelope will comprise less with higher luminosity. Therefore, the critical core mass, at which the mass of the core 
and the envelope becomes comparable, will become larger with higher luminosity.

\end{document}
