\documentclass[a4paper,12pt]{article}
\usepackage{graphicx, geometry, subfigure, amsmath, adjustbox, array}
\usepackage{xcolor}
\geometry{a4paper,left=2cm,right=2cm,top=1cm,bottom=2cm}
% \setlength{\baselineskip}{15pt}
\renewcommand{\baselinestretch}{1.5}
\renewcommand\arraystretch{1.5}
\renewcommand{\d}{\mathrm{d}}
\newcommand{\cm}{\mathrm{cm}}
\newcommand{\s}{\mathrm{s}}
\newcommand{\g}{\mathrm{g}}

\title{\textbf{Stars and Planets Problem Set6}}
\author{Qingru Hu}
\date{\today}

\begin{document}
\maketitle
\section*{\textbf{Exercise \uppercase\expandafter{\romannumeral6}.1 Synodical timescale}}
\subsection*{(a)}
The $t_{syn}$ is the least common multiple of $P_1$ and $P_2$:
\begin{align*}
    t_{syn} = [P_1, P_2]
\end{align*}

\subsection*{(b)}
\textcolor{red}{how to calculate?}


\section*{\textbf{Exercise \uppercase\expandafter{\romannumeral6}.2 Epicycle approximation}}
\subsection*{(a)}
Assuming that $e<<1$, from the Kelper equation $M = E - e \sin E$ we can have:
\begin{align*}
    \sin E &= \sin(M + e\sin E) = \sin M \cos(e\sin E) + \cos M \sin(e\sin E) \\
           &= \sin M (1+\frac{1}{2}(e\sin E)^2 +\cdots) + \cos M (e\sin E + \cdots) \\
           &= \sin M + \mathcal{O} (e)
\end{align*}
and:
\begin{align*}
    \cos E &= \cos(M + e\sin E) = \cos M \cos(e\sin E) - \sin M \sin(e\sin E) \\
           &= \cos M (1+\frac{1}{2}(e\sin E)^2 +\cdots) - \sin M (e\sin E + \cdots) \\
           &= \cos M - e\sin M \sin E + \mathcal{O} (e^2) = \cos M - e\sin ^2M + \mathcal{O} (e^2)\\
           &= \cos M + \mathcal{O} (e)
\end{align*}

Firstly consider $r$:
\begin{align*}
    \cos E &= e + \frac{r}{a} \cos \nu = \frac{e + \cos \nu}{1 + e\cos \nu} \\
    \cos \nu &= \frac{\cos E - e}{1 - e\cos E} \\
    r &= \frac{a(\cos E - e)}{\cos \nu} = a (1 - e \cos E) \\
\end{align*}
Plug in $\cos E = \cos M + \mathcal{O} (e)$ and we have:
\begin{align*}
    r \approx  a (1 - e \cos M) + \mathcal{O} (e^2)
\end{align*}

Secondly consider $\nu$:
\begin{align*}
    \cos \nu &= \frac{\cos E - e}{1 - e\cos E}  = (\cos E - e)(1 + e\cos E) = \cos E - e \sin ^2E + \mathcal{O} (e^2)\\
\end{align*}
Plug in $\sin E = \sin M + \mathcal{O} (e)$ and $\cos E = \cos M - e\sin ^2M + \mathcal{O} (e^2)$, and we have:
\begin{align*}
    \cos \nu = \cos M - 2e \sin ^2M + \mathcal{O} (e^2)
\end{align*}
If we assume that $\nu \approx M + 2e\sin M + \mathcal{O} (e^2)$, we can have:
\begin{align*}
    \cos \nu &= \cos (M + 2e\sin M + \mathcal{O} (e^2)) \\
            &= \cos M \cos(2e\sin M) - \sin M \sin(2e\sin M) + \mathcal{O} (e^2) \\
            &= \cos M - 2 e\sin ^2M + \mathcal{O} (e^2)
\end{align*}
Therefore, we can conclude that $\nu \approx M + 2e\sin M + \mathcal{O} (e^2)$.

\subsection*{(b)}
\textcolor{red}{how to prove?}

\subsection*{(c)}
\begin{align*}
    \phi_{\text{eff}}'(r_o) &= An r_o^{n-1} - \frac{l_z^2}{2} \frac{2}{r_o^3} = 0 \\
    r_o &= (\frac{l_z^2}{An})^{\frac{1}{n+2}}
\end{align*}

\subsection*{(d)}
Expanding the potential around $r = r_o$:
\begin{align*}
    \phi_{\text{eff}} = \phi_{\text{eff}}(r_o) + \phi_{\text{eff}}'(r_o) x + \frac{1}{2} \phi_{\text{eff}}''(r_o) x^2
\end{align*}
where $x = r - r_o$ and $\phi_{\text{eff}}'(r_o)=0$.
Therefore the equation of motion (3) becomes:
\begin{align*}
    \ddot{x} &= - x \phi_{\text{eff}}''(r_o) = -x(An(n-1)r_o^{n-2} + 3 l_z^2 r_o^{-4}) = -(n+2)l_z^2 r_o^{-4} x \\
\end{align*}
Compared to $\ddot{x} = -\kappa^2 x$, we have:
\begin{align*}
    \kappa &= \frac{\sqrt{n+2} l_z}{r_o^2} = \sqrt{n+2} (\frac{l_z^{\frac{2-n}{2}}}{An})^{-\frac{2}{n+2}}
\end{align*}
In a Keplerian potential (n=-1):
\begin{align*}
    \kappa = A^2 / l_z^3 = \Omega
\end{align*}
where $\Omega$ is the orbital frequency.
\textcolor{red}{why the orbital frequency}

\subsection*{(e)}
\textcolor{red}{For which values of n does the circular orbit solution become unstable? What is the physical
reason?}

\section*{\textbf{Exercise \uppercase\expandafter{\romannumeral6}.3 The Trojans}}
\section*{(a)}
From the law of cos, we have:
\begin{align*}
    r_1^2 &= m^2 + 2mr\cos \theta + r^2 \\
    r_2^2 &= (1-m)^2 - 2(1-m)r \cos \theta + r^2
\end{align*}
Because $m<<1$, we can expand $r_1^{-1}$ as:
\begin{align*}
    r_1^{-1} &= (m^2 + 2mr\cos \theta + r^2)^{-1/2} \\
            &\approx r^{-1} (1 + 2(m/r) \cos \theta + (m/r)^2)^{-1/2} \\
            &\approx r^{-1} (1 - (m/r) \cos \theta) \\
            &\approx (1-\Delta) (1 - (m/r) \cos \theta) \\
            &= 1 - (m/r) \cos \theta - \Delta \\
            &\approx 1 - m \cos \theta - \Delta
\end{align*}
We can expand $r_2^{-1}$ as:
\begin{align*}
    r_2^{-1} &= ((1-m)^2 - 2(1-m)r \cos \theta + r^2)^{-1/2} \\
            &\approx (1 - 2 \cos \theta + 1)^{-1/2} \\
            &= \frac{1}{\sqrt{2(1- \cos \theta)}}
\end{align*}
We can expand $r^2$ as:
\begin{align*}
    r^2 = (1+\Delta)^2 = 1 + 2\Delta + \Delta^2
\end{align*}
Plug the three above relations into the effective potential:
\begin{align*}
    \phi_{\text{eff}} &= -\frac{1-m}{r_1} - \frac{m}{r_2} - \frac{1}{2}r^2 \\
    \phi_{\text{eff}} &= m(\cos \theta - \frac{1}{\sqrt{2(1- \cos \theta)}}) - \frac{1}{2}\Delta^2 - \frac{3}{2}
\end{align*}

\section*{(b)}
\textcolor{red}{how to derive!}

\section*{(c)}

\section*{(d)}

\section*{(e)}

\section*{(f)}

\section*{(g)}

\section*{(h)}


\section*{\textbf{Exercise \uppercase\expandafter{\romannumeral6}.4 Tides}}
\section*{(a)}
For the Earth-Moon system, the value for n (the mean motion) is $n = \frac{2\pi}{28 \text{days}} = 0.22 \text{day}^{-1}$.

\section*{(b)}
The spin-down timescale is:
\begin{align*}
    t_{\mathrm{de}-\text { spin, } \mathrm{p}}^{-1}=\frac{\dot{\Omega}_p}{\Omega_p}=-\frac{\Gamma}{\Omega_p I_p}=-\frac{3 k_{2 p}}{2 Q C_I} \frac{m_s^2}{\left(m_s+m_p\right) m_p}\left(\frac{R_p}{d}\right)^3 \frac{n}{\Omega_p} n
\end{align*}
For the Moon to spin-down the Earth, the love number $k_{2p} = $, the Quality factor $Q = $, 
the inertia factor $C_I = $, $m_s = 7.3477\times 10^{25} \ \g$, $m_p =5.974\times 10^{27} \ \g $, $R_p = 6.378\times 10^8 \ \cm$, $d = 384,401 \text{km}$, $n = \frac{2\pi}{28 \text{days}}$, and $\Omega_P = \frac{2\pi}{1 \ \text{day}}$. 
Therefore the timescale for the Moon to spin down the Earth is 4.4 billion years .

For the Sun to spin-down the Earth, the love number $k_{2p} = 0.3$, the Quality factor $Q = 12$, 
the inertia factor $C_I = 0.33$, $m_s = 1.99\times 10^{33}\ \g$, $m_p =5.974\times 10^{27} \ \g $, $R_p = 6.378\times 10^8 \ \cm$, $d = 1 \ au$, $n = \frac{2\pi}{365 \text{days}}$, and $\Omega_P = \frac{2\pi}{1 \ \text{day}}$.
Therefore the timescale for the Sun to spin down the Earth is 19.9 billion years.

\section*{(c)}
The time taken for the Moon to "crash" into the Earth is:
\begin{align*}
    t_{\text{orbit}}^{-1} = \frac{9k}{2Q} \frac{m_s}{m_p} (\frac{R_p}{d})^5 n
\end{align*}
Plug in the values from (b) and we can get $t_{\text{orbit}} = 7 \ \text{Gyr}$.

\section*{(d)}
\textcolor{red}{how to motivate?}

\section*{\textbf{Exercise \uppercase\expandafter{\romannumeral6}.5 Hot Jupiter migration by tides}}
\section*{(a)}
The total energy of the original orbit is:
\begin{align*}
    E_0 = -\frac{Gm}{2a_0}
\end{align*}
where the $m$ is the total mass of the star and the hot jupiter.
The kinetic energy of the original orbit is:
\begin{align*}
    K_0 = \frac{Gm}{2a_0}
\end{align*}
The gravitational potential energy of the orbit is:
\begin{align*}
    U_0 = - \frac{Gm}{a_0}
\end{align*}
When the magnitude of the orbit velocity is suddenly changed by a factor of $f$, the kinetic energy changes to:
\begin{align*}
    K_1 = f^2 K_0 = \frac{f^2}{2} \frac{Gm}{a_0}
\end{align*}
The gravitational energy remains the same:
\begin{align*}
    U_1 = U_0 = - \frac{Gm}{a_0}
\end{align*}
Therefore, the total energy changes to:
\begin{align*}
    E_1 = U_1 + K_1 = (\frac{f^2}{2} - 1) \frac{Gm}{a_0}
\end{align*}
Compared to the energy expression of the Kelperian orbit $E_1 = - \frac{Gm}{2a_1}$ we can get:
\begin{align*}
    a_1 = \frac{a_0}{2 - f^2}
\end{align*}

When the magnitude of the orbit velocity is suddenly changed by a factor of $f$, the total angular momentum changes to:
\begin{align*}
    l_1 = fl_0 = f\sqrt{Gma_0(1-e_0^2)}
\end{align*}
Compared to the angular momentum expression of the Kelperian orbit $l_1 = \sqrt{Gma_1(1-e_1^2)}$ we can get:
\begin{align*}
    e_1 = 1 - f^2
\end{align*}

The pericenter $r_{p1}$ is:
\begin{align*}
    r_{p1} = a_1 (1 - e_1) = \frac{f^2}{2 - f^2} a_0
\end{align*}

\section*{(b)}
During this tidal dissipation step, the hot Jupiter's angular momentum is roughly conserved as it circularizes to a final, close-in
semimajor axis $e_2=0$:
\begin{align*}
    l_1 &= \sqrt{Gma_1(1-e_1^2)} = l_2 = \sqrt{Gma_2} \\
    a_2 &= a_1 (1-e_1^2) = a_1 f^2 (2 - f^2) = a_0 f^2
\end{align*}
If $a_0 = 5$ au and $a_2 = 0.05$ au, then $f = 0.1$.

\section*{(c)}
I don't think the existence of these planets can be explained by this mechanism.
\textcolor{red}{explain my answer}


\section*{\textbf{Exercise \uppercase\expandafter{\romannumeral6}.6 Geometry of resonances}}
\section*{(a)}
\section*{(b)}
\section*{(c)}
\section*{(d)}

\section*{\textbf{Exercise \uppercase\expandafter{\romannumeral6}.7 Planet trapping}}
\section*{(a)}
\section*{(b)}
\section*{(c)}
\section*{(d)}

\end{document}
