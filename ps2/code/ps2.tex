\documentclass[a4paper,12pt]{article}
\usepackage{graphicx, geometry, subfigure, amsmath, adjustbox, array}
\geometry{a4paper,left=2cm,right=2cm,top=1cm,bottom=2cm}
\setlength{\baselineskip}{12pt}
\renewcommand\arraystretch{1.5}
\renewcommand{\d}{\mathrm{d}}
\newcommand{\cm}{\mathrm{cm}}
\newcommand{\s}{\mathrm{s}}
\newcommand{\g}{\mathrm{g}}

\title{Stars and Planets Problem Set2}
\author{Qingru Hu}
\date{\today}

\begin{document}
\maketitle
\section*{\textbf{Exercise \uppercase\expandafter{\romannumeral1}.1 Radiation Pressure}}
\subsection*{(a)}
The dimension of $P_{\text{rad}}$ is $[E][L]^{-3}$. The physical quantities that are related to 
pressure from the derivation of state equations are:
\begin{align*}
    kT \sim [E] \\
    h \sim [E][T]
\end{align*}
However, only with the two quantities above we cannot cancel out $T$ and add in $L$. 
Given the pressure is produced by radiation, 
we can introduce the speed of light $c \sim  [L][T]^{-1}$ to help us.
After some calculations, we can get that the dimension of radiation pressure is:
\begin{equation*}
    P_{\text{rad}} \sim \frac{(kT)^4}{h^3c^3}
\end{equation*}

\subsection*{(b)}
From the energy density $u_\nu$, we can get the distribution function for the energy $f(\nu)$:
\begin{equation*}
    f(\nu) = u_\nu / (h\nu) = \frac{8\pi \nu^2}{c^3} \frac{1}{e^{h\nu/kT} - 1}
\end{equation*}
And the relation between the distribution function for the energy and the momenta is:
\begin{align*}
    f(\nu) \text{d}\nu &= f(p) \text{d}p \\
    p &= \frac{h}{\lambda} = \frac{h\nu}{c} \\
    \text{d}p &= \frac{h}{c} \text{d}\nu
\end{align*}
Therefore, the distribution function for the momenta is:
\begin{equation*}
    f(p) = \frac{8\pi}{h^3} p^2 \frac{1}{e^{cp/kT}-1}
\end{equation*}
The radiation pressure is:
\begin{equation*}
    P_{\text{rad}} = \int_0^{+\infty} c p f(p) \text{d} p = \frac{8\pi^5}{45} \frac{(kT)^4}{h^3 c^3}
\end{equation*}
The relation between radiation energy density and radiation pressure is:
\begin{equation*}
    P_\text{rad} = \frac{u_\text{rad}}{3}
\end{equation*}


\section*{\textbf{Exercise \uppercase\expandafter{\romannumeral1}.2 Physical structure of protoplanetary disks}}
\subsection*{(a)}
The standard astrophysical “cosmic” abundances for the elements gives:
\begin{equation*}
    X=0.75, \ Y= 0.25
\end{equation*}
Since the hydrogen and helium gas are in molecular form, the relative mean molecular weight is:
\begin{align*}
    \mu_{\text{H}_2} = 2 \\
    \mu_{\text{He}} = 4
\end{align*}
The mean molecular weight $\mu$ of the gas is:
\begin{align*}
    \frac{1}{\mu} &= \frac{X}{\mu_{\text{H}_2}} + \frac{Y}{\mu_{\text{He}}} \\
    \mu &= 2.29
\end{align*}

\subsection*{(b)}
Assume the mass of the star is $M$, the gravitational constant is $G$ and the positive direction of $z$ axis is pointing upward.
Write down Newton's second law in the vertical direction:
\begin{equation*}
    g_{\text{z}, \star} = - \frac{GM}{r^2 + z^2} \frac{z}{\sqrt{r^2 + z^2}} = - \frac{GM}{r^3} \frac{z}{(1 + (\frac{z}{r})^2)^{\frac{3}{2}}}
\end{equation*}
For $z << r$:
\begin{equation*}
    \frac{z}{(1 + (\frac{z}{r})^2)^{\frac{3}{2}}} = z(1-\frac{3}{2} (\frac{z}{r})^2) \sim z
\end{equation*}
Therefore:
\begin{equation*}
    g_{\text{z}, \star} \approx - \frac{GM}{r^3} z = - \Omega^2_\text{K} z
\end{equation*}
where
\begin{equation*}
    \Omega_\text{K} = \sqrt{\frac{GM}{r^3}}
\end{equation*}

\subsection*{(c)}
The equation of state for the gas in the disk is:
\begin{equation*}
    P = nkT = \frac{\rho}{\mu m_u} kT
\end{equation*}
Since the $g_{\text{z}, \star}$ has included the minus sign, the hydrostatic balance is:
\begin{align*}
    \frac{\text{d} P}{\text{d} z} = \rho g_{\text{z}, \star}
\end{align*}
Combine the two equations and we get the ordinary differential equation for $\rho(z)$:
\begin{equation*}
    \frac{\text{d} \rho}{\text{d} z} = - \frac{\Omega^2_\text{K} \mu m_u}{kT} z \rho
\end{equation*}
The solution is:
\begin{equation*}
    \rho(z) = C e^{-\frac{\Omega^2_\text{K} \mu m_u}{2kT} z^2}
\end{equation*}

\subsection*{(d)}
The standard deviation $\sigma$ or the scaleheight $H$ is:
\begin{equation*}
    \sigma = H = \sqrt{\frac{kT}{\mu m_u \Omega^2_\text{K}}} = \sqrt{\frac{kT}{\mu m_u}} = \sqrt{\frac{kT}{\mu m_u G M}} r^{\frac{3}{2}}
\end{equation*}
And the $\rho(z)$ can be written as:
\begin{equation*}
    \rho(z) = C e^{-\frac{z^2}{2 H^2}}
\end{equation*}

\subsection*{(e)}
The surface density is:
\begin{align*}
    \Sigma &= \int_{-\infty}^{+\infty} \rho(z) \text{d}z = \int_{-\infty}^{+\infty} C e^{-\frac{z^2}{2 H^2}} \text{d}z \\
    \Sigma &= C H \sqrt{2\pi} \\
    C &= \frac{\Sigma}{H \sqrt{2\pi}}
\end{align*}
Therefore, $\rho(z)$ can be rewritten as:
\begin{equation*}
    \rho(z) = \frac{\Sigma}{H \sqrt{2\pi}} e^{-\frac{z^2}{2 H^2}}
\end{equation*}

\subsection*{(f)}
When $z<<r$, use the small angle approximation:
\begin{equation*}
    \tan \theta = \frac{z}{r} \approx \theta
\end{equation*}
Since $z$ only depends on $r$, take derivatives on both sides:
\begin{equation*}
    \frac{\text{d} \theta}{\text{d} r} = \frac{\text{d} }{\text{d} r} (\frac{z}{r})
\end{equation*}
The fraction $\epsilon$ of the stellar luminosity $L_\star$ that is being intercepted 
by the disk at radius interval $[r, r + \Delta r]$ is:
\begin{align*}
    \epsilon &= \frac{1}{2\pi} \int \cos \theta \text{d} \theta \text{d} \phi 
    = \int_{\theta}^{\theta + \text{d} \theta} \cos \theta \text{d} \theta 
    = \int_{r}^{r + \text{d} r} \frac{\text{d} }{\mathrm{d} r} (\frac{z}{r}) \text{d}r \\
    &= \frac{\text{d} }{\mathrm{d} r} (\frac{z}{r}) \Delta r
    = \beta \frac{\text{d} }{\mathrm{d} r} (\frac{H}{r}) \Delta r
\end{align*}
Pulge in $\frac{H}{r} = \sqrt{\frac{k}{\mu m_u G M}} T^{\frac{1}{2}} r^{\frac{1}{2}}$ and denote $\alpha = \sqrt{\frac{k}{\mu m_u G M}}$:
\begin{align*}
    \epsilon = \beta \alpha \frac{\d }{\d r} (\sqrt{T r}) \Delta r
\end{align*}



\subsection*{(g)}
The radiation is in turn intercepted at the disk midplane($z=0$). For the disk at radius interval $[r, r + \Delta r]$, 
the radiation area is $2\pi r \Delta r$. The heating and cooling radiation equation reads:
\begin{align*}
    \frac{1}{2} L_\star \epsilon &= 2 \times 2\pi r \Delta r \sigma T^4 \\
    \frac{\d T}{\d r} &= \frac{16 \pi \sigma}{L_\star \beta \alpha} r^{\frac{1}{2}} T^{\frac{9}{2}} - \frac{T}{r} \\
\end{align*}
Since we don not have additional initial or boundary conditions, so we can guess that $T(r) = C r^{n}$.
Pluge the assumption into the above differential equation we can get:
\begin{align*}
    Cn r^{n-1} = \frac{16 \pi \sigma}{L_\star \beta \alpha} r^{\frac{1}{2}} (Cr^n)^{\frac{9}{2}} - C r^{n-1}
\end{align*}
This equation must be true for every r, so:
\begin{align*}
    Cn &= \frac{16 \pi \sigma}{L_\star \beta \alpha} C^{\frac{9}{2}} - C \\
    n - 1 &= \frac{9}{2} n + \frac{1}{2}
\end{align*}
We get:
\begin{align*}
    C &= (\frac{L_\star \beta \alpha}{28 \pi \sigma})^{\frac{2}{7}}\\
    n &= -\frac{3}{7}
\end{align*}
Therefore:
\begin{equation*}
    T(r) = (\frac{L_\star \beta \alpha}{28 \pi \sigma})^{\frac{2}{7}} r^{-\frac{3}{7}}
\end{equation*}


%T(r) = (\frac{L_\star \beta^2 \alpha^2}{8\pi \sigma} \frac{1}{r})^{\frac{1}{3}}
%T^4 = \frac{L_\star \beta^2}{4\pi \sigma} \frac{H}{r^2} (\frac{H}{r})' \\
%T(r) = (\frac{L_\star \beta^2}{4\pi \sigma} \frac{H}{r^2} (\frac{H}{r})')^{\frac{1}{4}}

\subsection*{(h)}
Given $L_\star = 1 \ L_{\bigodot}$, $M = 1 \ M_{\bigodot}$, $r = 1\ au$ and $\beta =3$:
\begin{align*}
    T(r) = 123.94 \ \text{K}
\end{align*}

\subsection*{(i)}
If the photons instead are absorbed along the way, 
the midplane temperature would become lower.
If the photons are absorbed along the way,
less radiation energy will reach the midplane. 
According to the heating and cooling radiation balance, 
less energy will be emittted by the midplane, so the 
midplane temperature will be lower.

\section*{\textbf{Exercise \uppercase\expandafter{\romannumeral1}.3 A solar-mass White Dwarf}}
\subsection*{(a)}
The equation of state for degenerate non-relativistic electron gas is:
\begin{align*}
    P &= (\frac{3}{\pi})^{\frac{2}{3}} \frac{h^2}{20 m_e} (\frac{\rho}{\mu_e m_u})^{\frac{5}{3}} 
    = K_{\text{e, d, non}} \rho^{1 + \frac{2}{3}}\\
    K_{\text{e, d, non}} &= (\frac{3}{\pi})^{\frac{2}{3}} \frac{h^2}{20 m_e} (\frac{1}{\mu_e m_u})^{\frac{5}{3}} \\
    n &= \frac{3}{2} 
\end{align*}
From properties of polytropes we can get:
\begin{align*}
    K &= N_n G M^{\frac{n-1}{n}} R^{\frac{3-n}{n}} \\
    K_{\text{e, d, non}} &= N_{\frac{3}{2}} G M^{\frac{1}{3}} R \\
    R &= \frac{K_{\text{e, d, non}}}{N_{\frac{3}{2}} G} M^{-\frac{1}{3}}
\end{align*}
where $N_{\frac{3}{2}} = 0.424$, $\mu_e = 2$ and $M = 1 \ M_{\bigodot}$. Pluge in the 
concrete numbers and we get:
\begin{equation*}
    R = 1.39 \ R_{\bigoplus}
\end{equation*}

\subsection*{(b)}
From properties of polytropes we can get:
\begin{equation*}
    \rho_c = r_n \bar{\rho} = r_n \frac{M}{\frac{4}{3} \pi R^3}
\end{equation*}
Given $n = \frac{3}{2}$ and $r_n = 5.99$, we can get:
\begin{equation*}
    \rho_c = 4.060\times 10^6 \ \g \ \cm^{-3}
\end{equation*}

\subsection*{(c)}
The electrons at the center are approaching relativistic motions. 
Given the electrons in the White Dwarf is degenerate, 
the De Broglie wavelength of the electron $\lambda_e = \frac{h}{p} = \frac{h}{m_e v_e}$
is almost equal to the distance between electrons $d = (\frac{1}{n_e})^{-\frac{1}{3}}$. 
We assume that the electrons at the center are non-relativistic and the momenta is 
$p = m_e v_e$, we have:
\begin{align*}
    \lambda_e &= d \\
    v_e &= \frac{h}{m_e} (n_e)^{\frac{1}{3}} = \frac{h}{m_e} (\frac{\rho_c}{\mu_e m_u})^{\frac{1}{3}} \\
    v_e &= 7.778\times 10^{10} \ \cm \ \s^{-1}
\end{align*}
If the electrons at the center are non-relativistic, the velocity of the electrons is greater than 
the velocity of light $c=3\times 10^{10}\ \cm \ \s^{-1} $.
Therefore, our assumption is not right, that is the electrons at the center are approaching relativistic motions.

\section*{\textbf{Exercise \uppercase\expandafter{\romannumeral1}.4 Mass-radius relationship for solid planets}}
\subsection*{(a)}
The $n=1$ solution for the Lane-Emden equation is:
\begin{equation*}
    \phi_1 = \frac{\sin \xi}{\xi}
\end{equation*}
From the outer boundary conditions, we can get:
\begin{align*}
    \phi_1(R) &= \frac{\rho(r=R)}{\rho_c} = \frac{\rho_0}{\rho_c} \\
    \phi_1(R) &= \frac{\sin \xi_R}{\xi_R} \\
    \rho_c &= \rho_0 \frac{\xi_R}{\sin \xi_R}
\end{align*}
The mass of the planet can be got from integration:
\begin{align*}
    M &= \int_{0}^{\xi_R} 4\pi \rho r^2 \d r = 4\pi \rho_c \lambda_1^3 (-\xi_R^2 \phi_1'(\xi_R)) \\
    &= 4\pi \lambda_1^3 \rho_0 \frac{\xi_R}{\sin \xi_R} (\sin \xi_R - \xi_R\cos \xi_R)
\end{align*}

\subsection*{(b)}
For $\xi_R << 1$:
\begin{align*}
    \frac{\sin \xi_R - \xi_R\cos \xi_R}{\sin \xi_R}
    &= 1 - \xi_R \frac{\cos \xi_R}{\sin \xi_R}
    = 1 - \xi_R \frac{1-\frac{1}{2} \xi_R^2}{\xi_R - \frac{1}{6} \xi_R^3} \\
    &= 1 - \frac{1-\frac{1}{2} \xi_R^2}{1-\frac{1}{6} \xi_R^2}
    = 1 - (1-\frac{1}{2} \xi_R^2)(1+\frac{1}{6} \xi_R^2) \\
    &= \frac{1}{3} \xi_R^2 + \frac{1}{12} \xi_R^4 \approx \frac{1}{3} \xi_R^2
\end{align*}
Pluge the above approximation into the formula of $M$ and notice $\lambda_1 = \frac{R}{\xi_R}$:
\begin{equation*}
    M = 4\pi \lambda_1^3 \rho_0 \xi_R \frac{1}{3} \xi_R^2 = \frac{4\pi}{3} \rho_0 R^3
\end{equation*}

\subsection*{(c)}
Given that $M = 1 \ M_{\bigoplus}$, $\rho_0 = 3 \ \g \ \cm^{-3}$ and $\lambda_1 = \frac{R_{\bigoplus}}{\xi_R}$, 
use Mathematica to solve the equation numerically:
\begin{align*}
    M_{\bigoplus} &= 4 \pi \rho_0 (\frac{R_{\bigoplus}}{\xi_R})^3 \frac{\xi_R}{\sin \xi_R} (\sin \xi_R - \xi_R \cos \xi_R) \\
    \xi_R &= 2.368 \\
    \lambda_1 &= \frac{R_{\bigoplus}}{\xi_R} = 0.42 \ R_{\bigoplus}
\end{align*}

\subsection*{(d)}
When $n=1$, $\lambda_1$ is a constant, not related to the central density:
\begin{align*}
    \lambda_1 = \frac{(n+1) K}{4\pi G} \rho_c^{\frac{1}{n} - 1} = \frac{2 K}{4\pi G} = 0.42 \ R_{\bigoplus}
\end{align*}
From the relation:
\begin{align*}
    \rho_c &= \rho_0 \frac{\xi_R}{\sin \xi_R} \\
    \xi_R &= \frac{R}{\lambda_1} 
\end{align*}
We can take $\xi_R (R)$ as a function of $\rho_c$ and find the maximum $\xi_R (R)$.
Denote $y=\xi_R$ and $x = \rho_c$ and differentiate $x$ on both sides:
\begin{align*}
    \d(x \sin y) = \d(\rho_0 y) \\
    \frac{\d y}{\d x} = \frac{\sin y}{\rho_0 - x\cos y}
\end{align*}
When $y=\pi$, the first derivative equals zero, and the maximum radius is:
\begin{equation*}
    R_{\text{max}} = 1.32 \ R_{\bigoplus}
\end{equation*}

\subsection*{(e)}
The polytropic index n will be higher than 1. The bigger the radius, the softer the equation of state, 
the smaller the adiabatic index $\gamma = 1 + \frac{1}{n}$, and the higher the $n$.


\section*{\textbf{Exercise \uppercase\expandafter{\romannumeral1}.5 The Gamow peak}}
\subsection*{(a)}
Take the derivative of $f(E)$ and set it to zero:
\begin{align*}
    f'(E_0) = \frac{\pi}{2} \frac{E_*^{\frac{1}{2}}}{E_0^{\frac{3}{2}}} - \frac{1}{kT}\\
    E_0 = (\frac{1}{4} \pi^2 E_* (kT)^2)^{\frac{1}{3}}
\end{align*}

\subsection*{(b)}
At the peak of $f(E)$, the first order derivative of $f(E)$ is zero, so the $(E - E_0)$ term disappears.

\subsection*{(c)}
\begin{align*}
    A &= f(E_0) = (\frac{\pi^2 E_*}{4kT})^{\frac{1}{3}} \\
    B &= \frac{1}{2} f''(E_0) = -\frac{3\pi}{8} \frac{E_*^{\frac{1}{2}}}{E_0^{\frac{5}{2}}}
     = -\frac{3\pi}{8} \frac{E_*^{\frac{1}{2}}}{(\frac{1}{4} \pi^2 E_* (kT)^2)^{\frac{5}{6}}}
\end{align*}
When $E$ deviate from $E_0$, the $\exp((E-E_0)^2)$ decreases rapidly, so we can integrate $E$ from $-\infty$ to $+\infty$ to get $r_{12}$:
\begin{align*}
    r_{12} &= n_1 n_2 \frac{8}{\pi m_\mu} (kT)^{-\frac{3}{2}} S(E_0) \int \exp(f(E_0) + \frac{1}{2} f''(E_0) (E-E_0)^2) \d E\\
    &= n_1 n_2 \frac{8}{\pi m_\mu} (kT)^{-\frac{3}{2}} S(E_0) \int \exp(A + B (E-E_0)^2) \d E\\
    &\approx n_1 n_2 \frac{8}{\pi m_\mu} (kT)^{-\frac{3}{2}} S(E_0) e^A \frac{1}{\sqrt{-B}} \int_{-\infty}^{+\infty} \exp(- (\sqrt{-B}(E-E_0))^2) \d \sqrt{-B} E\\
    &= n_1 n_2 \frac{8}{\pi m_\mu} (kT)^{-\frac{3}{2}} S(E_0) e^A \sqrt{\frac{\pi}{-B}} \\
    r_{12}&\approx \frac{8n_1 n_2}{m_\mu \sqrt{- \pi B}} (kT)^{-\frac{3}{2}} S(E_0) e^A
\end{align*}

\subsection*{(d)}
Therefore:
\begin{align*}
    \gamma = \frac{\d \log r_{12}}{\d \log T} = A - \frac{3}{2} + \frac{5}{6} = -\frac{2}{3} + (\frac{\pi^2 E_*}{4kT})^{\frac{1}{3}}
\end{align*}

\subsection*{(e)}
For the pp-chain, $Z_1 = Z_2 = 1$, $m_\mu = 1/2 m_u$:
\begin{equation*}
    \gamma_{\text{pp-chain}} = 4.55
\end{equation*}

For the CNO cycle, $Z_1 = 1$, $Z_2 = 7$, $m_\mu = 14/15 m_u $:
\begin{equation*}
    \gamma_{\text{CNO cycle}} = 22.84
\end{equation*}


\end{document}
