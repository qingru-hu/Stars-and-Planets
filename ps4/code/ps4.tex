\documentclass[a4paper,12pt]{article}
\usepackage{graphicx, geometry, subfigure, amsmath, adjustbox, array}
\usepackage{xcolor}
\geometry{a4paper,left=2cm,right=2cm,top=1cm,bottom=2cm}
\setlength{\baselineskip}{12pt}
\renewcommand\arraystretch{1.5}
\renewcommand{\d}{\mathrm{d}}
\newcommand{\cm}{\mathrm{cm}}
\newcommand{\s}{\mathrm{s}}
\newcommand{\g}{\mathrm{g}}

\title{\textbf{Stars and Planets Problem Set4}}
\author{Qingru Hu}
\date{\today}

\begin{document}
\maketitle
\section*{\textbf{Exercise \uppercase\expandafter{\romannumeral4}.1 Photon diffusion}}
\subsection*{(a)}
The mean free path of the distribution $\bar{s}$ is:
\begin{align*}
    \bar{s} = \frac{\int_{0}^{+\infty} s l^{-1} e^{-s/l} \d s}{\int_{0}^{+\infty} l^{-1} e^{-s/l} \d s} = 
    \frac{l \int_{0}^{+\infty} x e^{-x} \d x}{\int_{0}^{+\infty} e^{-x} \d x}
    = l \ (\text{where } x = s/l)
\end{align*}


\subsection*{(b)}
The distribution makes the escape time longer. The probability of the step size falling between $s$, $s+\d s$ 
decreases exponentially as the step size increases. The probability of the step size smaller than the mean free path $l$
is much more higher than that of the step size bigger than the mean free path, so the escape time will become longer.

\subsection*{(c)}
I think the escape time will become shorter. If there exists density gradient, diffusion will come into the picture.
Diffusion will help photons to move from dense core to sparse surface, making the escape time shorter.



\section*{\textbf{Exercise \uppercase\expandafter{\romannumeral4}.2 Ledoux stability criterion}}
\subsection*{(a)}

\subsection*{(b)}


\subsection*{(c)}


\subsection*{(d)}


\subsection*{(f)}



\end{document}
