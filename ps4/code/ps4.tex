\documentclass[a4paper,12pt]{article}
\usepackage{graphicx, geometry, subfigure, amsmath, adjustbox, array}
\usepackage{xcolor}
\geometry{a4paper,left=2cm,right=2cm,top=1cm,bottom=2cm}
\setlength{\baselineskip}{12pt}
\renewcommand\arraystretch{1.5}
\renewcommand{\d}{\mathrm{d}}
\newcommand{\cm}{\mathrm{cm}}
\newcommand{\s}{\mathrm{s}}
\newcommand{\g}{\mathrm{g}}

\title{\textbf{Stars and Planets Problem Set4}}
\author{Qingru Hu}
\date{\today}

\begin{document}
\maketitle
\section*{\textbf{Exercise \uppercase\expandafter{\romannumeral4}.1 Photon diffusion}}
\subsection*{(a)}
The mean free path of the distribution $\bar{s}$ is:
\begin{align*}
    \bar{s} = \frac{\int_{0}^{+\infty} s l^{-1} e^{-s/l} \d s}{\int_{0}^{+\infty} l^{-1} e^{-s/l} \d s} = 
    \frac{l \int_{0}^{+\infty} x e^{-x} \d x}{\int_{0}^{+\infty} e^{-x} \d x}
    = l \ (\text{where } x = s/l)
\end{align*}


\subsection*{(b)}
The distribution makes the escape time longer. The probability of the step size falling between $s$, $s+\d s$ 
decreases exponentially as the step size increases. The probability of the step size smaller than the mean free path $l$
is much more higher than that of the step size bigger than the mean free path, so the escape time will become longer.

\subsection*{(c)}
I think the escape time will become shorter. If there exists density gradient, diffusion will come into the picture.
Diffusion will help photons to move from dense core to sparse surface, making the escape time shorter.



\section*{\textbf{Exercise \uppercase\expandafter{\romannumeral4}.2 Ledoux stability criterion}}
\subsection*{(a)}
The expression for the two gradients are:
\begin{align*}
    \nabla &= \frac{\d \log T}{\d \log P} \\
    \nabla_\mu &= \frac{\d \log \mu }{\d \log P}
\end{align*}
Divide both sides of Equation (3) by $\d P/ p = \d \log P$:
\begin{align*}
    1 &= \chi_\rho \frac{\d \log \rho}{\d \log P} + \chi_T \frac{\d \log T}{\d \log P} + \chi_\mu \frac{\d \log \mu}{\d \log P} \\
    1 &= \chi_\rho \frac{\d \log \rho}{\d \log P} + \chi_T \nabla + \chi_\mu \nabla_\mu
\end{align*}

\subsection*{(b)}
Since for an adiabatic gas the composition is the same, $\nabla_\mu = 0$. 
For constant entropy process, $\nabla = \nabla_\text{ad}$ and $P\propto \rho^{\gamma_{\text{ad}}}, \d \log P/\d \log \rho = \gamma_\text{ad}$.
Equation(3) becomes:
\begin{align*}
    1 &= \chi_\rho / \gamma_\text{ad} + \chi_T \nabla_\text{ad} \\
    \nabla_\text{ad} &= (1-\chi_\rho / \gamma_\text{ad})/\chi_T
\end{align*}

\subsection*{(c)}
For an ideal mono-atomic gas, the equation of state is $P = \frac{\rho kT}{\mu m_u}$ and $\gamma_\text{ad} = 5/3$. Besides, 
$\chi_\rho = 1$ and $\chi_T = 1$.
\begin{align*}
    \nabla_\text{ad} &= (1-\chi_\rho / \gamma_\text{ad})/\chi_T = 2/5
\end{align*}

\subsection*{(d)}
From Equation(4) we can get:
\begin{align*}
    \frac{\d \log \rho}{\d \log P} = (1 - \chi_T \nabla - \chi_\mu \nabla_\mu)/\chi_\rho
\end{align*}
From section (b) we can get:
\begin{align*}
    \frac{1}{\gamma_\text{ad}} = (1 - \chi_T \nabla_\text{ad})/\chi_\rho
\end{align*}
Th e instability criterion can lead to the condition for convection:
\begin{align*}
    \frac{\d \log \rho}{\d \log P} &< \frac{1}{\gamma_\text{ad}} \\
    (1 - \chi_T \nabla - \chi_\mu \nabla_\mu)/\chi_\rho &< (1 - \chi_T \nabla_\text{ad})/\chi_\rho \\
    \nabla > \nabla_\text{ad} -& \frac{\chi_\mu}{\chi_T}\nabla_\mu
\end{align*}



\section*{\textbf{Exercise \uppercase\expandafter{\romannumeral4}.3 Structure of the Sun}}
\subsection*{(a)}
The relation between the mean molecular mass $\mu$ and the fractional radius is shown as below.
\begin{figure}[htbp]
    \centering
    \includegraphics*[width=8cm]{mu.png}
\end{figure}

\subsection*{(b)}
The two dashed lines in the above figure shows the 20\% and 95\% fractional radius. 
We all know that $\mu_H = 1/2$, $\mu_{He}=4/3$ and $\mu_{Z} = 2$. In the central $20\%$, 
the core of the sun is mostly composed of Helium, so the mean molecular weight is between 1/2 and 
4/3; the interior 20\%-95\% region is mostly composed of Hydrogen, so the mean molecular weight is 
near 1/2; in the outer 5\% region, metals have been brought into this shell due to convection, so the 
mean molecular weight increase sharply in the outer shell.

\end{document}
