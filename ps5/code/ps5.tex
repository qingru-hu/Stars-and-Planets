\documentclass[a4paper,12pt]{article}
\usepackage{graphicx, geometry, subfigure, amsmath, adjustbox, array}
\usepackage{xcolor}
\geometry{a4paper,left=2cm,right=2cm,top=1cm,bottom=2cm}
% \setlength{\baselineskip}{12pt}
\renewcommand{\baselinestretch}{1.5}
\renewcommand{\d}{\mathrm{d}}
\newcommand{\cm}{\mathrm{cm}}
\newcommand{\s}{\mathrm{s}}
\newcommand{\g}{\mathrm{g}}
\newcommand{\todo}{\color{red}}

\title{\textbf{Stars and Planets Problem Set5}}
\author{Qingru Hu}
\date{\today}

\begin{document}
\maketitle
\section*{\textbf{Exercise \uppercase\expandafter{\romannumeral5}.1 Solar photosphere}}
\subsection*{(a)}
The Saha equation for the ionization of hydrogen is:
\begin{align*}
    \frac{x^2}{1-x} = \frac{e^{-13.6eV/kT}}{n_H} (\frac{2\pi kT m_e}{h^2})^{3/2}
\end{align*}
where $x = n_{\text{ H \uppercase\expandafter{\romannumeral2}}} / n_{\text{ H \uppercase\expandafter{\romannumeral1}}}$ is 
the ionization fraction.

Assume the temperature of the Sun's photosphere is $T=5780K$ and $n_H = \frac{\rho}{m_u}$:
\begin{align*}
    \frac{x^2}{1-x} &= 8.14 \times 10^{-9} \\
    x & \approx 0
\end{align*}
Therefore, hydrogen in the solar photosphere is mostly neutral.

\subsection*{(b)}
% The Sun's photosphere has an effective temperature of about $T_{eff}=5780 K$ and a density of $\rho = 3 \times 10^{-7} \g/\cm^3$.

% Although the average thermal energy of the photospheric gas (about 0.5 eV per atom) and the average photon energy of the photospheric radiation (about 2 eV) is lower than the dissociation energy of molecular hydrogen 4.5eV, 
% but there still exist large amounts of photospheric gas and radiation in the high-energy end of the distribution function to dissociate the molecular hydrogen. 
% Therefore, hydrogen molecules that form in the photosphere will quickly be destroyed by interactions with high energy gas or photons.

% Besides, there are not many efficient ways to form hydrogen molecules.
% Another possible way to form molecular hydrogen is by three-body reactions, 
% such as H + H + H -> H2 + H or H + H + He -> H2 + He, 
% but these reactions are very slow and require high densities. 
% The density of the photosphere is too low for these reactions to be significant.

% Therefore, molecular hydrogen is virtually nonexistent in the Sun's photosphere due to its high dissociation rate and low formation rate.
Although the average thermal energy of the photospheric gas (about 0.5 eV per atom) and the average photon energy of the photospheric radiation (about 2 eV) is lower than the dissociation energy of molecular hydrogen 4.5eV, 
the high-energy photons produced by the nuclear fusion in the core will dissociate the molecular hydrogen when they pass through the photosphere.

\subsection*{(c)}
The net reaction of the combination of hydrogen atoms into molecules is: H + H -> H2.
Assuming the internal partition functions are all 1, the Saha eqaution writes:
\begin{align*}
    \frac{n_{H, atom} n_{H, atom}}{n_{H2}} = (\frac{2\pi kT}{h^2} \frac{m_u}{2})^{3/2} e^{-4.48eV/kT}
\end{align*}
If molecular hydrogen makes up 1\% of the hydrogen, then $n_{H, atom} = 0.99 n_H$, $n_{H2} = 0.01 n_H$ and $n_H = \rho/m_u$.
We can solve the above equation numerically and get a temperature of $T=3800\ K$.

Sunspots should become cooler than 3800K before molecular hydrogen will make 
up more than 1\% of the hydrogen.

\section*{\textbf{Exercise \uppercase\expandafter{\romannumeral5}.2 Rotational emission with CO}}
\subsection*{(a)}
The moment of inertia of the CO molecule $I$ is:
\begin{align*}
    I &= m_C r_C^2 + m_O r_O^2 \\
      &= 12 m_u (8a_0/7)^2 + 16 m_u (6a_0 /7)^2 \\
      &= 192m_u a_0^2/7 
\end{align*}
where $r$ is the distance to the center of mass.
The rotational constant $B_e$ is:
\begin{align*}
    B_e = \frac{\hbar^2}{2I} = \frac{7}{384}\frac{\hbar^2}{m_ua_0^2} = 3.16 K
\end{align*}

\subsection*{(b)}
Since $A_{ul}^{-1}$ is the expectation time of the excited state, the probability of an atom at the excited state 
falling back to the lower state in the unit time interval 1s is:
\begin{align*}
    P_{ul, 1s} = \frac{1}{A_{ul}^{-1}}
\end{align*}
The expectation number of the excited atoms that will fall back in the unit time interval is:
\begin{align*}
    N_{ul, 1s} = \frac{N_u}{A_{ul}^{-1}}
\end{align*}
The released energy in the unit time interval is:
\begin{align*}
    E_{ul, 1s} = \frac{N_u h \nu_{ul}}{A_{ul}^{-1}} = N_u h \nu_{ul} A_{ul}
\end{align*}
The energy flux per unit time per unit solid angle is just:
\begin{align*}
    F_{ul} = \frac{N_u h \nu_{ul} A_{ul}}{4\pi}
\end{align*}

\subsection*{(c)}
According to the Boltzman distribution:
\begin{align*}
    \frac{N_u}{N} &= \frac{g_u}{Z} e^{-\frac{E_u}{kT}} \\
    \frac{N_u}{N g_u} &= \frac{1}{Z} e^{-\frac{E_u}{kT}} \\
    \log (\frac{N_u}{N g_u}) &= -\log Z - \frac{E_u}{kT}
\end{align*}
The plot for $J=0, 1, \cdots, 10$ and $T=10, 20, 30K$ is shown as below.

\begin{figure}[htbp]
    \centering
    \includegraphics*[width=10cm]{rotation.png}
\end{figure}

 By comparing the intensity of transition lines between certain energies, Angie can obtain the temperature of the gas.
 From Equation(2) we can see that, only $N_u$ in the flux $F_{ul}$ is related to the temperature. At different temperatures, 
 the fractions of $N_u$ to $N$ are different, resulting in different intensity of the same transition line.


\subsection*{(d)}
From the plot we can see that, the fraction of $N_u$ to $N$ is almost the same for all three temperatures at $J_u=1$ and $J_u = 2$. 
If Angie only observes low level J lines ($J_u=1$ and $J_u = 2$), the fluxes at which the cloud radiates in the CO line are 
almost the same for all three temperature. Therefore, she couldn't digtinguish between different temperatures.

At high Js, the fraction of $N_u$ to $N$ is very small, even $10^{-25}$ for $J_u=8$ and $T=10K$. Thus, the transition line flux 
for low temperatures are too small, resulting in large error in determining temperatures.

\section*{\textbf{Exercise \uppercase\expandafter{\romannumeral5}.3 Mars and Titan}}
\subsection*{(a)}
If we ignore the greenhouse effect, we can calculate the 
surface temperature from the equilibrium of radiation:
\begin{align*}
    T_{eq} = T_{eff, \star} (1-a)^{1/4} (\frac{R_\star}{2d})^{1/2} = 210.42 K
\end{align*}
where $a=0.25$ $T_{eff, \star}=5780K$, $R_\star = 6.96 \times 10^{10} \cm$ and $d=1.52$au.

\subsection*{(b)}
The hydrostatic balance and the ideal gas law are:
\begin{align*}
    -\frac{\d P}{\d z} &= \rho \frac{GM}{r^2} \\
    P &= \frac{\rho kT}{\mu m_u}
\end{align*}
The pressure scaleheight of the atmosphere is:
\begin{align*}
    H = -\frac{P}{\d P/ \d z} = \frac{kT}{\mu m_u} \frac{r^2}{GM}
\end{align*}
For Mars, $T = T_{eq} = 210.42 K$, $r = 3.397\times10^8 \cm$, 
$M = 0.11 M_{\bigoplus}$ and $\mu=44$, so the scaleheight is $H = 10.46 \text{km}$.

\subsection*{(c)}
For Mars, $P=0.00628 \ \text{atm} = 636.3 \ Pa = 6363 \ Ba$ and 
$T=T_{eq} = 210.42 K$, we have the density of the atmosphere of Mars $\rho$:
\begin{align*}
    \rho = \frac{P \mu m_u}{kT} = 1.6 \times 10^{-5} \ \g / \cm^{-3}
\end{align*}

The total mass of the CO2 atmosphere $M$ is:
\begin{align*}
    M = 4\pi r^2 H \rho = 2.43 \times 10^{19} \g
\end{align*}

\subsection*{(d)}
When the Sun ascends the RGB, it will reach a peak luminosity of 
about 2300 times its current value and a radius of about 170 times its current value. 
According to the equilibrium of radiation, the surface temperature of 
Titan will increase significantly, which will cause Titan's atmosphere 
to heat up and expand, possibly lossing some of its mass to space.
It could also trigger more complex chemical reactions in Titan's atmosphere, 
producing more organic compounds. After receiving more radiation when the Sun ascends the RGB, Titan's 
atmosphere may become hotter, thinner and richer in compounds, 
and not suitable for human settlement.

\section*{\textbf{Exercise \uppercase\expandafter{\romannumeral5}.4 Habitable Zone}}
\subsection*{(a)}
Assuming the simple greenhouse model:
\begin{align*}
    T_s &= (1+\tau_{IR})^{1/4} T_{irr} \\
    T_{irr} &= T_{eff, \star} (1-a)^{1/4} (\frac{R_\star}{2d})^{1/2} \\
\end{align*}
For the Earth, $T_s \propto d^{-1/2}$.
If the Earth is at 0.95 au, $T_{s, max} = T_s 0.95^{-1/2} = 295.5\ K =22.5 \ ^\circ C$.
If the Earth is at 1.37 au, $T_{s, max} = T_s 1.37^{-1/2} = 246.1\ K = -26.9 \ ^\circ C$.
Kasting's values correspond to the surface temperature range $-26.9 \ ^\circ C \leq T_s \leq 22.5 \ ^\circ C$.
\subsection*{(b)}
The equilibrium temperature of Venus $T_{irr}$ is:
\begin{align*}
    T_{irr} &= T_{eff, \star} (1-a)^{1/4} (\frac{R_\star}{2d})^{1/2} = 227.52 \ K
\end{align*}
where $a=0.77$, $T_{eff, \star}=5780K$, $R_\star = 6.96 \times 10^{10} \cm$ and $d=0.72$au.
Assuming a simple greenhouse model:
\begin{align*}
    T_s &= (1+\tau_{IR})^{1/4} T_{irr} \\
    \tau_{IR} &= (T_s / T_{irr})^4 - 1 = 109.1
\end{align*}
Therefore, the IR optical depth of Venus atmosphere is $\tau_{IR} = 109.1$.

\subsection*{(c)}
Combing the radiation equilibrium and the greenhouse model, we can 
get an expression for $d$:
\begin{align*}
    d = \frac{R_\star}{2} ((1+\tau)(1-a))^{1/2} (T_{eff, \star} / T_s)^2
\end{align*}
For a surface temperature $T_s = 295.5 \ K$, the distance to the sun is $d=4.48$au;
for a surface temperature $T_s = 246.1 \ K$, the distance to the sun is $d=6.46$au.

Therefore, the range of the habitable zone for Venus is $4.48 \ \text{au} \leq d \leq 6.46 \ \text{au}$.


\end{document}
