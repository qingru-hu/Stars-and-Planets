\documentclass[a4paper,12pt]{article}
\usepackage{graphicx, geometry, subfigure, amsmath, adjustbox, array}
\usepackage{xcolor}
\geometry{a4paper,left=2cm,right=2cm,top=1cm,bottom=2cm}
% \setlength{\baselineskip}{15pt}
\renewcommand{\baselinestretch}{1.5}
\renewcommand\arraystretch{1.5}
\renewcommand{\d}{\mathrm{d}}
\newcommand{\cm}{\mathrm{cm}}
\newcommand{\s}{\mathrm{s}}
\newcommand{\g}{\mathrm{g}}

\title{\textbf{Stars and Planets Problem Set6}}
\author{Qingru Hu}
\date{\today}

\begin{document}
\maketitle
\section*{\textbf{Exercise \uppercase\expandafter{\romannumeral6}.1 Synodical timescale}}
\subsection*{(a)}
Assuming $P_1<P_2$, the synodical timescale satisfies:
\begin{align*}
    (\omega_1 - \omega_2) t_{\text{syn}} &= 2\pi \\
    (2\pi/P_1 - 2\pi/P_2) t_{\text{syn}} &= 2\pi \\
\end{align*}
Therefore:
\begin{align*}
    t_{\text{syn}} &= \frac{1}{1/P_1 - 1/P_2} = \frac{P_1 P_2}{P_2 - P_1}
\end{align*}

\subsection*{(b)}
According to the third Kelpler law:
\begin{align*}
    \frac{a_1^3}{P_1^2} = \frac{a_2^3}{P_2^2} = \frac{Gm}{4\pi^2}
\end{align*}
Expand $P_1/P_2$ in terms of $b/a_2<<1$ ($b = a_2 - a_1, b/a_1<<1$):
\begin{align*}
    \frac{P_1}{P_2} = (\frac{a_1}{a_2})^{3/2} = (1 - \frac{b}{a_2})^{3/2} = 1 - \frac{3}{2} \frac{b}{a_2}
\end{align*}
Plug into $t_{\text{syn}}$ and we have:
\begin{align*}
    t_{\text{syn}} &= P_1\frac{1}{1 - P_1/P_2} =\frac{2}{3} \frac{a_2}{b} P_1
\end{align*}


\section*{\textbf{Exercise \uppercase\expandafter{\romannumeral6}.2 Epicycle approximation}}
\subsection*{(a)}
Assuming that $e<<1$, from the Kelper equation $M = E - e \sin E$ we can have:
\begin{align*}
    \sin E &= \sin(M + e\sin E) = \sin M \cos(e\sin E) + \cos M \sin(e\sin E) \\
           &= \sin M (1+\frac{1}{2}(e\sin E)^2 +\cdots) + \cos M (e\sin E + \cdots) \\
           &= \sin M + \mathcal{O} (e)
\end{align*}
and:
\begin{align*}
    \cos E &= \cos(M + e\sin E) = \cos M \cos(e\sin E) - \sin M \sin(e\sin E) \\
           &= \cos M (1+\frac{1}{2}(e\sin E)^2 +\cdots) - \sin M (e\sin E + \cdots) \\
           &= \cos M - e\sin M \sin E + \mathcal{O} (e^2) = \cos M - e\sin ^2M + \mathcal{O} (e^2)\\
           &= \cos M + \mathcal{O} (e)
\end{align*}

Firstly consider $r$:
\begin{align*}
    \cos E &= e + \frac{r}{a} \cos \nu = \frac{e + \cos \nu}{1 + e\cos \nu} \\
    \cos \nu &= \frac{\cos E - e}{1 - e\cos E} \\
    r &= \frac{a(\cos E - e)}{\cos \nu} = a (1 - e \cos E) \\
\end{align*}
Plug in $\cos E = \cos M + \mathcal{O} (e)$ and we have:
\begin{align*}
    r \approx  a (1 - e \cos M) + \mathcal{O} (e^2)
\end{align*}

Secondly consider $\nu$:
\begin{align*}
    \cos \nu &= \frac{\cos E - e}{1 - e\cos E}  = (\cos E - e)(1 + e\cos E) = \cos E - e \sin ^2E + \mathcal{O} (e^2)\\
\end{align*}
Plug in $\sin E = \sin M + \mathcal{O} (e)$ and $\cos E = \cos M - e\sin ^2M + \mathcal{O} (e^2)$, and we have:
\begin{align*}
    \cos \nu = \cos M - 2e \sin ^2M + \mathcal{O} (e^2)
\end{align*}
If we assume that $\nu \approx M + 2e\sin M + \mathcal{O} (e^2)$, we can have:
\begin{align*}
    \cos \nu &= \cos (M + 2e\sin M + \mathcal{O} (e^2)) \\
            &= \cos M \cos(2e\sin M) - \sin M \sin(2e\sin M) + \mathcal{O} (e^2) \\
            &= \cos M - 2 e\sin ^2M + \mathcal{O} (e^2)
\end{align*}
Therefore, we can conclude that $\nu \approx M + 2e\sin M + \mathcal{O} (e^2)$.

\subsection*{(b)}
In the polar coordinate, the acceleration is:
\begin{align*}
    \boldsymbol{\ddot{r}} = (\ddot{r} - r \dot{\theta}^2) \hat{r} + (r\ddot{\theta} + 2\dot{r}\dot{\theta}) \hat{\theta}
\end{align*}
We consider the $\ddot{r}$:
\begin{align*}
    \ddot{r} = \boldsymbol{\ddot{r}}\cdot \hat{r} + r \dot{\theta}^2
\end{align*}
We have:
\begin{align*}
    \boldsymbol{\ddot{r}}\cdot \hat{r} &= -\nabla \phi \cdot \hat{r} = - \frac{\partial \phi}{\partial r} = - A n r^{n-1} \\
    r \dot{\theta}^2 &= (r^2 \dot \theta)^2/r^3 = l_z^2 / r^3 \\
    \ddot{r} &= - A n r^{n-1} + l_z^2 / r^3 = -\frac{\partial}{\partial r}(Ar^n + \frac{l_z^2}{2r^2}) =  -\frac{\partial \phi_{\text{eff}}}{\partial r}
\end{align*}
where $\phi_{\text{eff}} = \phi(r) + \frac{l_z^2}{2r^2}$ is the effective potential.

\subsection*{(c)}
\begin{align*}
    \phi_{\text{eff}}'(r_o) &= An r_o^{n-1} - \frac{l_z^2}{2} \frac{2}{r_o^3} = 0 \\
    r_o &= (\frac{l_z^2}{An})^{\frac{1}{n+2}}
\end{align*}

\subsection*{(d)}
Expanding the potential around $r = r_o$:
\begin{align*}
    \phi_{\text{eff}} = \phi_{\text{eff}}(r_o) + \phi_{\text{eff}}'(r_o) x + \frac{1}{2} \phi_{\text{eff}}''(r_o) x^2
\end{align*}
where $x = r - r_o$ and $\phi_{\text{eff}}'(r_o)=0$.
Therefore the equation of motion (3) becomes:
\begin{align*}
    \ddot{x} &= - x \phi_{\text{eff}}''(r_o) = -x(An(n-1)r_o^{n-2} + 3 l_z^2 r_o^{-4}) = -(n+2)l_z^2 r_o^{-4} x \\
\end{align*}
Compared to $\ddot{x} = -\kappa^2 x$, we have:
\begin{align*}
    \kappa &= \frac{\sqrt{n+2} l_z}{r_o^2} = \sqrt{n+2} (\frac{l_z^{\frac{2-n}{2}}}{An})^{-\frac{2}{n+2}}
\end{align*}
In a Keplerian potential (n=-1):
\begin{align*}
    \kappa = A^2 / l_z^3 = \frac{(GM)^2}{(\sqrt{GM a})^3} = \sqrt{\frac{GM}{a^3}} = \Omega
\end{align*}
where $\Omega$ is the keplerian orbital frequency.

\subsection*{(e)}
For $n<-2$ the circular orbit solution becomes unstable. 

If $n<-2$ the equation of motion will become:
\begin{align*}
    \ddot{x} &= -(n+2)l_z^2 r_o^{-4} x = \kappa^2 x\\
    \kappa &> 0
\end{align*}
The general solution for this differential equation is either $x(t) = x_0 e^{kt}$ 
or $x(t) = x_0 e^{-kt}$, which indicates that the second object will be scattered to 
infinity or collide into the primary object.  

\section*{\textbf{Exercise \uppercase\expandafter{\romannumeral6}.3 The Trojans}}
\section*{(a)}
From the law of cos, we have:
\begin{align*}
    r_1^2 &= m^2 + 2mr\cos \theta + r^2 \\
    r_2^2 &= (1-m)^2 - 2(1-m)r \cos \theta + r^2
\end{align*}
Because $m<<1$, we can expand $r_1^{-1}$ as:
\begin{align*}
    r_1^{-1} &= (m^2 + 2mr\cos \theta + r^2)^{-1/2} \\
            &\approx (1 + 2\Delta + \Delta^2 + 2m\cos \theta)^{-1/2} \\
            &\approx 1 - 1/2 (2\Delta + \Delta^2 + 2m\cos \theta) + 3/8 (2\Delta + \Delta^2 + 2m\cos \theta)^2 \\
            &\approx 1 - \Delta + \Delta^2 -m \cos \theta
\end{align*}
We can expand $r_2^{-1}$ as:
\begin{align*}
    r_2^{-1} &= ((1-m)^2 - 2(1-m)r \cos \theta + r^2)^{-1/2} \\
            &\approx (1 - 2 \cos \theta + 1)^{-1/2} \\
            &= \frac{1}{\sqrt{2(1- \cos \theta)}}
\end{align*}
We can expand $r^2$ as:
\begin{align*}
    r^2 = (1+\Delta)^2 = 1 + 2\Delta + \Delta^2
\end{align*}
Plug the three above relations into the effective potential:
\begin{align*}
    \phi_{\text{eff}} &= -\frac{1-m}{r_1} - \frac{m}{r_2} - \frac{1}{2}r^2 \\
    \phi_{\text{eff}} &= m(\cos \theta - \frac{1}{\sqrt{2(1- \cos \theta)}}) - \frac{3}{2}\Delta^2 +m - 3/2
\end{align*}
Ignore the constants in the potential and we get:
\begin{align*}
    \phi_{\text{eff}} &= m(\cos \theta - \frac{1}{\sqrt{2(1- \cos \theta)}}) - \frac{3}{2}\Delta^2
\end{align*}

\section*{(b)}
The vector form of the equation of motion is:
\begin{align*}
    \ddot{\boldsymbol{r}} + 2(\boldsymbol{\omega} \times \dot{\boldsymbol{r}}) = - \nabla \phi_{\text{eff}}
\end{align*}
The radial component of the equation of motion is:
\begin{align*}
    \ddot{r} - r\dot \theta^2 -2 r \dot{\theta}= - \frac{\partial \phi_{\text{eff}}}{\partial r}
\end{align*}
Plug in the expression of the effective potential and we have:
\begin{align*}
    \ddot{\Delta} - (1 + \Delta) \dot{\theta}^2 - 2(1 + \Delta)\dot{\theta} = 3\Delta
\end{align*}

\section*{(c)}
Assume $\Delta << 1$ and we get:
\begin{align*}
    \ddot{\Delta} - \dot{\theta}^2 - 2\dot{\theta} = 3\Delta
\end{align*}
And $\dot{\theta}^2$ must be the higher order small value compared to $\dot{\theta}$. 
Therefore, the above equation can be reduced to:
\begin{align*}
    \ddot{\Delta} - 2\dot{\theta} = 3\Delta
\end{align*}
If the objects are confined in a small range of radius, then the motion 
along the radial direction can not be too large, so $\ddot{\Delta}$ is also a 
higher order small value. Equation (8) can be reduced to:
\begin{align*}
    2\dot{\theta} + 3\Delta = 0
\end{align*}

\section*{(d)}
The azimuthal component of the equation of motion is:
\begin{align*}
    r\ddot{\theta} + 2\dot{r}\dot{\theta} +2 \dot{r}= - \frac{1}{r}\frac{\partial \phi_{\text{eff}}}{\partial \theta}
\end{align*}
Plug in $r = 1 + \Delta$ and we have:
\begin{align*}
    (1 + \Delta) \ddot{\theta} + 2 \dot{\Delta} \dot{\theta} + 2 \dot{\Delta} = -\frac{1}{1+\Delta} \frac{\partial \phi_{\text{eff}}}{\partial \theta}
\end{align*}
Ignore the higher order small value:
\begin{align*}
    \ddot{\theta} + 2 \dot{\Delta} = -\frac{\partial \phi_{\text{eff}}}{\partial \theta}
\end{align*}

\section*{(e)}
The effective potential can be rewritten as:
\begin{align*}
    \phi_{\text{eff}} =-\frac{m}{2}(4\sin ^2 \frac{\theta}{2} +\frac{1}{\sin \theta/2}) - \frac{3}{2} \Delta^2 + m
\end{align*}
From the radial component of the equation of motion we have $\Delta = -2/3 \dot{\theta}$.
Plug the above two expressions into the azimuthal component and we can have:
\begin{align*}
    - \frac{1}{3} \ddot{\theta} = \frac{m}{2} \frac{\partial}{\partial \theta} (4\sin ^2 \frac{\theta}{2} +\frac{1}{\sin \theta/2}) + 3 \Delta \frac{\partial \Delta}{\partial \theta}
\end{align*}
Ignore the higher order small value $3 \Delta \frac{\partial \Delta}{\partial \theta}$ and times $\dot{\theta}$ on both sides:
\begin{align*}
    - \dot{\theta} \ddot{\theta} &= \frac{3}{2} m \frac{\partial}{\partial \theta} (4\sin ^2 \frac{\theta}{2} +\frac{1}{\sin \theta/2}) \dot{\theta} \\
    0 &= \frac{\d}{\d t}(\frac{1}{2} \dot{\theta}^2 + \frac{3}{2}m (4\sin ^2 \frac{\theta}{2} +\frac{1}{\sin \theta/2}))
\end{align*}
The integral of motion under this approximation gives a conserved quantity $I$:
\begin{align*}
    I = \frac{1}{2} \dot{\theta}^2 + \frac{3}{2}m (4\sin ^2 \frac{\theta}{2} +\frac{1}{\sin \theta/2})
\end{align*}

\section*{(f)}
The potential component is:
\begin{align*}
    U = \frac{3}{2}m (4\sin ^2 \frac{\theta}{2} +\frac{1}{\sin \theta/2})
\end{align*}
Take the first derivative of the potential and we can get the Lagrange points:
\begin{align*}
    U'/m &= 3/2(4\sin \frac{\theta}{2} \cos \frac{\theta}{2} - \frac{1}{2} \frac{\cos \theta/2}{\sin^2 \theta/2}) \\
    \cos \frac{\theta}{2} &= 0 \ \text{or} \ \sin \frac{\theta}{2} = 1/2 \\
    \theta &= \pi/3, \pi/2 \ \text{or} \ 5\pi/3
\end{align*}
The potential and the Lagrange points are shown as below.
\begin{figure*}[htbp]
    \centering
    \includegraphics*[width=10cm]{Ls.png}
\end{figure*}

\section*{(g)}
The widest possible Trojan orbit extend from L4 to near L3. By solving the inequality 
$U<U(L_3)$ we can have:
\begin{align*}
    \theta\in [2\arcsin \frac{\sqrt{2} - 1}{2}, \pi]
\end{align*}

L4 is a local minimum so the asteroids can stay motionless around L4 $\dot{\theta(L_4)} = 0$.
From the conserved quantity $I$ we can have:
\begin{align*}
    \frac{1}{2} \dot{\theta(L_3)}^2 + U(\theta(L_3)) &= \frac{1}{2} \dot{\theta(L_4)}^2 + U(\theta(L_4)) \\
    \dot{\theta(L_3)} &= \sqrt{2 (U(\theta(L_4)) - U(\theta(L_3)))} = \sqrt{3m}
\end{align*}
Terefore:
\begin{align*}
    2\dot{\theta} +3 \Delta &= 0 \\
    \Delta &\in [-2\sqrt{\frac{m}{3}}, 2\frac{m}{3}]
\end{align*}
the total radial width of these Trojans is:
\begin{align*}
    b-a = 4\sqrt{\frac{m}{3}} = 4 \times \sqrt{\frac{9.5 \times 10^{-4}}{3}} = 0.37 \ \text{au}
\end{align*}

\section*{(h)}
Expand the potential to the second order at L4 $\theta=\pi/3$:
\begin{align*}
    U''/m &= 3/2(2\cos \theta + \frac{1}{2} \sin^{-3} \frac{\theta}{2} \cos ^2 \frac{\theta}{2} + \frac{1}{4} \sin ^{-1} \frac{\theta}{2}) \\
    \omega_{\text{lib}} &= \sqrt{k/m} = \sqrt{U''/m} = \sqrt{4.5} = \frac{3\sqrt{3}}{2} \ \omega \\
    t_{\text{lib}} &= \frac{2\pi}{\omega_{\text{lib}}} = \frac{1}{2.12}\frac{2\pi}{\omega} = 0.47 t
\end{align*}
The orbital period for Jupiter is $t=12$ years, so the orbital period for Jupiter's Trojans around L4 is $t_{\text{lib}}=5.6$ years.
\textcolor{red}{not correct}

\section*{\textbf{Exercise \uppercase\expandafter{\romannumeral6}.4 Tides}}
\section*{(a)}
For the Earth-Moon system, the value for n (the mean motion) is $n = \frac{2\pi}{28 \text{days}} = 0.22 \text{day}^{-1}$.

\section*{(b)}
The spin-down timescale is:
\begin{align*}
    t_{\mathrm{de}-\text { spin, } \mathrm{p}}^{-1}=\frac{\dot{\Omega}_p}{\Omega_p}=-\frac{\Gamma}{\Omega_p I_p}=-\frac{3 k_{2 p}}{2 Q C_I} \frac{m_s^2}{\left(m_s+m_p\right) m_p}\left(\frac{R_p}{d}\right)^3 \frac{n}{\Omega_p} n
\end{align*}
For the Moon to spin-down the Earth, the love number $k_{2p} = $, the Quality factor $Q = $, 
the inertia factor $C_I = $, $m_s = 7.3477\times 10^{25} \ \g$, $m_p =5.974\times 10^{27} \ \g $, $R_p = 6.378\times 10^8 \ \cm$, $d = 384,401 \text{km}$, $n = \frac{2\pi}{28 \text{days}}$, and $\Omega_P = \frac{2\pi}{1 \ \text{day}}$. 
Therefore the timescale for the Moon to spin down the Earth is 4.4 billion years .

For the Sun to spin-down the Earth, the love number $k_{2p} = 0.3$, the Quality factor $Q = 12$, 
the inertia factor $C_I = 0.33$, $m_s = 1.99\times 10^{33}\ \g$, $m_p =5.974\times 10^{27} \ \g $, $R_p = 6.378\times 10^8 \ \cm$, $d = 1 \ au$, $n = \frac{2\pi}{365 \text{days}}$, and $\Omega_P = \frac{2\pi}{1 \ \text{day}}$.
Therefore the timescale for the Sun to spin down the Earth is 19.9 billion years.

\section*{(c)}
The time taken for the Moon to "crash" into the Earth is:
\begin{align*}
    t_{\text{orbit}}^{-1} = \frac{9k}{2Q} \frac{m_s}{m_p} (\frac{R_p}{d})^5 n
\end{align*}
Plug in the values from (b) and we can get $t_{\text{orbit}} = 7 \ \text{Gyr}$.

\section*{(d)}
The collision velocity would have been similar to their mutual surface escape velocity 
$v_{\text{esc}} = \sqrt{2G(m_1 + m_2)/(R_1 + R_2)}$. The typical physical properties 
that are included in the tidal effects are just $m_{1,2}$ and $R_{1,2}$, so from analysis of the dimension, 
the collision velocity can only be in the form of $\sim \sqrt{G(m_1 + m_2)/(R_1 + R_2)}$, which can only 
be different from the mutual surface escape velocity of a constant (that is, they are of the same magnitude).

\section*{\textbf{Exercise \uppercase\expandafter{\romannumeral6}.5 Hot Jupiter migration by tides}}
\section*{(a)}
The total energy of the original orbit is:
\begin{align*}
    E_0 = -\frac{Gm}{2a_0}
\end{align*}
where the $m$ is the total mass of the star and the hot jupiter.
The kinetic energy of the original orbit is:
\begin{align*}
    K_0 = \frac{Gm}{2a_0}
\end{align*}
The gravitational potential energy of the orbit is:
\begin{align*}
    U_0 = - \frac{Gm}{a_0}
\end{align*}
When the magnitude of the orbit velocity is suddenly changed by a factor of $f$, the kinetic energy changes to:
\begin{align*}
    K_1 = f^2 K_0 = \frac{f^2}{2} \frac{Gm}{a_0}
\end{align*}
The gravitational energy remains the same:
\begin{align*}
    U_1 = U_0 = - \frac{Gm}{a_0}
\end{align*}
Therefore, the total energy changes to:
\begin{align*}
    E_1 = U_1 + K_1 = (\frac{f^2}{2} - 1) \frac{Gm}{a_0}
\end{align*}
Compared to the energy expression of the Kelperian orbit $E_1 = - \frac{Gm}{2a_1}$ we can get:
\begin{align*}
    a_1 = \frac{a_0}{2 - f^2}
\end{align*}

When the magnitude of the orbit velocity is suddenly changed by a factor of $f$, the total angular momentum changes to:
\begin{align*}
    l_1 = fl_0 = f\sqrt{Gma_0(1-e_0^2)}
\end{align*}
Compared to the angular momentum expression of the Kelperian orbit $l_1 = \sqrt{Gma_1(1-e_1^2)}$ we can get:
\begin{align*}
    e_1 = 1 - f^2
\end{align*}

The pericenter $r_{p1}$ is:
\begin{align*}
    r_{p1} = a_1 (1 - e_1) = \frac{f^2}{2 - f^2} a_0
\end{align*}

\section*{(b)}
During this tidal dissipation step, the hot Jupiter's angular momentum is roughly conserved as it circularizes to a final, close-in
semimajor axis $e_2=0$:
\begin{align*}
    l_1 &= \sqrt{Gma_1(1-e_1^2)} = l_2 = \sqrt{Gma_2} \\
    a_2 &= a_1 (1-e_1^2) = a_1 f^2 (2 - f^2) = a_0 f^2
\end{align*}
If $a_0 = 5$ au and $a_2 = 0.05$ au, then $f = 0.1$.

\section*{(c)}
I don't think the existence of these planets can be explained by this mechanism.

A planet's initial periapse is approximately half of their final orbit 
radius if the palnet is perturbed onto a highly eccentric orbit $e>>1$:
\begin{align*}
    a_2 = a_1(1-e_1^2) \approx 2a_1(1-e_1) = 2 r_{p1} 
\end{align*}
The periapse must be larger than 1 Roche radius of the planet to survive the tidal 
disruption of its host star. Therefore, we expect to see surviving planets beyond 
2 Roche radius if the planet is formed through the high eccentricity tidal migration.
% The tidal force rasied by the star on the hot jupiters is too strong at 1-2 Roche radii that it will tear apart hot jupiters on high eccentric orbits.


\section*{\textbf{Exercise \uppercase\expandafter{\romannumeral6}.6 Geometry of resonances}}
\section*{(a)} Interactions after the conjunction are stronger, because the two planets are closer after the conjunction.
\section*{(b)} This results in a net negative torque on planet 1, because planet 1 moves ahead of planet 2 and is dragged down by planet 2.
\section*{(c)} It causes the next conjunction point to be closer to pericenter, because the inner planet 1 loses angular momentum and the outer planet 2 gains angular momentum.
\section*{(d)} Resonances near pericenter are therefore stable.

\section*{\textbf{Exercise \uppercase\expandafter{\romannumeral6}.7 Planet trapping}}
\section*{(a)}
Combine the steady-state solutions of the resonance forcing equation for semi-major axis and 
eccentricity:
\begin{align*}
    \dot{a} &= 2(j+\delta_1) G_e^j q_1 n_1 a e \sin \phi_{\text{eq}} - \frac{a}{t_a} = 0\\
    \dot{e} &= G_e^j q_1 n_1 \sin \phi_{\text{eq}} - \frac{e}{t_e} = 0
\end{align*}
Solve the above two equations and we get:
\begin{align*}
    \sin\phi_{\text{eq}} &= \frac{1}{G_e^j q_1 n_1} \frac{1}{\sqrt{2(j+\delta_1)t_e t_a}}\\
    e_{\text{eq}} &= \frac{\sqrt{t_e}}{\sqrt{2(j+\delta_1)t_a}}
\end{align*}
Since $G_e^j$ are negative for an internal perturber, $\sin \phi_{\text{eq}}$ is negative according to the equation of $e$.
And for an inner circular orbit and an outer eccentric orbit, the stable resonance is at apocenter, so 
$\cos \phi_{\text{eq}}$ should also be negative. The point (x, y) is in the third quadrant.

\section*{(b)}
For the inner perturber, when in a 3:2 resonance $j=2$, $G_e^j = -1.66$ and $\delta_1 = 1$.
Assuming the inner perturber is an 10 Earth-mass planet orbiting a solar-mass star at 1 au, 
$q_1 = 10m_{\bigoplus}/m_{\bigodot}$ and $n_1 = 2\pi/1 $yr.
The equilibrium eccentricity is $e_{\text{eq}} = 0.004$ and the equilibrium resonance angle is 
$\phi_{\text{eq}} = -187.5 ^{\circ}$.

\section*{(c)}
The equation for $\dot{\phi_{\text{res}}}$ is:
\begin{align*}
    \dot{\phi_{\text{res}}} &= - \Delta n_{\text{res}} + G_e^j \frac{q_1 n_1 \cos \phi_{\text{res}}}{e} \\
    \dot{\phi_{\text{res}}} &= - jn_2 \Delta + G_e^j \frac{q_1 n_1 \cos \phi_{\text{res}}}{e} \\
\end{align*}
where $\Delta n_{\text{res}} = jn_1 - (j+1) n_2$ and $\Delta \equiv \frac{P_2}{P_1} - \frac{j+1}{j}$.

Set $\dot{\phi_{\text{res}}}=0$ and assume that $n_2 = 2/3 n_1$:
\begin{align*}
    \Delta = G_e^j \frac{q_1 n_1 \cos \phi_{\text{res}}}{ejn_2} = 0.01
\end{align*}

\section*{(d)}
\textcolor{red}{what???}

\end{document}
